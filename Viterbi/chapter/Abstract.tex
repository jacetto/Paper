\begin{abstract}

Robot localization while navigating is one of the fundamental problems of robot navigation. 


!!say something about odometry error and loop closure!!


Appearance-based place recognition characterizes the map in nodes either in terms of global or local features extracted from sensor inputs. Following the local features approach, more precisely using the \textit{bag-of-word} scheme (\textit{BoW}) and its extension called \textit{Vocabulary-Tree} we present in this paper two contributions to the recognition pipeline. First a weak geometrical check that weights the \textit{BoW} matching. An observation on the problem definition together with the use of a dynamic programing algorithm allow for a fast assertion of basic geometrical characteristics while comparing local features of two 2D range data. Secondly a document augmentation of the database as initially proposed by \cite{Turcot} using directly the graph provided by the SLAM algorithm. Adding these two complements to a classical \textit{BoW} algorithm permits an increase in the retrieval performance while still allowing an online construction of the database.

We evaluate this method on three SLAM benchmark datasets and demonstrate that it achieves performances similare to state-of-the-art both in terms of recall and precision.


sdsds


\end{abstract}